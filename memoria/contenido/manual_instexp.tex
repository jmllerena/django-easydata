% ------------------------------------------------------------------------------
% Este fichero es parte de la plantilla LaTeX para la realización de Proyectos
% Final de Grado, protegido bajo los términos de la licencia GFDL.
% Para más información, la licencia completa viene incluida en el
% fichero fdl-1.3.tex

% Copyright (C) 2012 SPI-FM. Universidad de Cádiz
% ------------------------------------------------------------------------------

A continuación, se detallan la instrucciones para la correcta instalación y
explotación de la aplicación EasyData/Django para la apertura de datos en
proyectos Django.

% Las instrucciones de instalación y explotación del sistema se detallan a
% continuación.

\section{Introducción}

El software EasyData/Django, ofrecerá al usuario que incorpore la aplicación a
su proyecto facilidades a la hora de la publicación de los datos de su base de
datos, pudiendo elegir únicamente aquellos que estime conveniente el usuario,
haciendo uso de las diferentes ontologías disponibles en internet (como pueden
ser, Schema, FOAF, etc.), y en diferentes formatos ampliamente utilizados (como
pueden ser RDF, RDFa o microdata).

De esta forma, las principales funcionalidades que ofrece el software al usuario
son las siguientes:
\begin{itemize}
    \item Captación de los distintos modelos y fields de los mismos, de los que
        se encuentra compuesto el proyecto Django donde se instale la aplicación
        EasyData/Django.
    \item Carga de diferentes namespaces en formato RDF disponibles en internet.
    \item Configuración de los datos que serán visibles y aquellos que no se
        considerarán privados y no se mostrarán al exterior.
    \item Correspondencia de los modelos y fields del proyecto con los distintos
        elementos de las ontologías que se hayan cargado en el proyecto.
    \item Publicación de los datos en los diferentes formatos disponibles.
\end{itemize}

\section{Requisitos previos}

Para el correcto funcionamiento de la aplicación EasyData/Django, los únicos
requisitos hardware existentes son los propios de cualquier aplicación Python y
proyecto Django.

Por otro lado, en referencia a los requisitos software, como requisito base será
necesario un sistema operativo el cual permita le ejecución de código Python y
que cumpla con los requisitos software para la ejecución proyectos Django.
Además, como requisitos específicos existen los siguientes:
\begin{itemize}
    \item Al tratarse de una aplicación para proyectos Django, será necesario
        que el proyecto donde se vaya a incluir la aplicación EasyData/Django
        funcione bajo una versión igual o superior a la 1.4 de Django.
    \item Así mismo, también será necesario que se encuentre instalado el
        paquete \textit{rdflib} de Python (a ser posible en su versión 4.0.1, ya
        que no se ha probado para versiones anteriores y no se puede asegurar su
        correcto funcionamiento), ya que esta aplicación hace uso de él para
        ofrecer la mayoría de sus funcionalidades descritas anteriormente.
    \item Además, para la generación de los gráficos mediante Graphviz de la
        configuración de los modelos del proyecto, será necesario el paquete
        pydot.
\end{itemize}

\section{Inventario de componentes}

Junto con la aplicación EasyData/Django se incluyen los siguientes paquetes
software necesarios para el correcto funcionamiento de la aplicación:
\begin{itemize}
    \item Framework jQuery en su versión 1.9.1.
    \item Plugin jQuery-UI para el framework jQuery en su versión 1.10.3.
    \item Framework CSS Bootstrap en su versión 3.0.0.
\end{itemize}


\section{Procedimientos de instalación}
\label{sec:procinsta}

A continuación se detallan cada uno de los pasos necesarios para la correcta
instalación y explotación de la aplicación EasyData/Django dentro de un proyecto
Django.

Lo primero que hay que hacer es instalar los paquetes necesarios para que la
aplicación EasyData/Django funcione dentro de nuestro sistema, por lo que
deberemos de instalar/actualizar la versión de Django de nuestro sistema a una
versión igual o superior a la 1.4 y asegurarnos que nuestro proyecto Django es
compatible con esta versión. Lo primero podemos hacerlo mediante el comando pip:

\begin{lstlisting}[frame=L, language=bash, basicstyle=\footnotesize]
$> pip install django>=1.4
\end{lstlisting}

O bien, entrando en la web oficial de Django y descargando los ficheros fuente
del repositorio de Django y siguiendo las instrucciones de instalación. Esto se
encuentra correctamente explicado en la
\href{https://www.djangoproject.com/download/}{documentación oficial} de Django.

Para la instalación del paquete rdflib de Python, podremos proceder exactamente
de la misma forma, instalando el mismo desde pip de la siguiente forma:

\begin{lstlisting}[frame=L, language=bash, basicstyle=\footnotesize]
$> pip install rdflib==4.0.1
\end{lstlisting}

O bien desde el repositorio oficial GIT del proyecto, el cual se puede encontrar
en \href{https://github.com/RDFLib/rdflib}{GitHub} disponible para su descarga y
perfectamente documentado.

De igual forma, para la instalación del paquete pydot de Python, podemos hacerlo
a través de la herramienta pip de la siguiente forma:

\begin{lstlisting}[frame=L, language=bash, basicstyle=\footnotesize]
$> pip install pydot
\end{lstlisting}

Para la instalación de los paquetes comentados anteriormente, también podemos
hacer uso de la herramienta easy\_install, la cual es bastante similar a pip, de
la siguiente forma:

\begin{lstlisting}[frame=L, language=bash, basicstyle=\footnotesize]
$> easy_install django
$> easy_install rdflib
$> easy_install pydot
\end{lstlisting}

Una vez hemos instalado los paquetes necesarios para la ejecución de la
aplicación EasyData/Django, únicamente nos faltará descargar la aplicación
EasyData/Django y copiar el contenido de la aplicación dentro de nuestro
\textit{PYTHONPATH} o dentro del propio directorio del proyecto Django, junto
con el resto de aplicaciones que componen a dicho proyecto Django.

También existe la posibilidad de descargar la aplicación EasyData/Django desde
el Python Package Index, como hemos hecho con las aplicaciones anteriores de las
que hace uso el proyecto. Además, la aplicación EasyData/Django tiene
especificados los requisitos necesarios para su ejecución, por lo que al
instalarla desde el pip, el propio instalador se encargará de instalar las
aplicaciones de las que depende, ahorrándonos los pasos anteriores. Para ello
procederemos de igual forma que en los casos anteriores introduciendo uno de los
siguientes comandos:

\begin{lstlisting}[frame=L, language=bash, basicstyle=\footnotesize]
$> easy_install django-easydata
\end{lstlisting}

\begin{lstlisting}[frame=L, language=bash, basicstyle=\footnotesize]
$> pip install django-easydata
\end{lstlisting}

Una vez tengamos la aplicación EasyData/Django, el siguiente paso será incluir
dicha aplicación dentro del proyecto Django donde queremos hacer uso de la
misma. Para ello, solo tendremos que añadir la aplicación en el settings.py de
nuestro proyecto dentro de la variable \mbox{INSTALLED\_APPS}, quedando de la
siguiente forma:

\begin{lstlisting}[frame=L, language=Python, basicstyle=\footnotesize]
INSTALLED_APPS = (
    'django.contrib.auth',
    'django.contrib.contenttypes',
    'django.contrib.sessions',
    'django.contrib.sites',
    'django.contrib.messages',
    'django.contrib.staticfiles',
    'django.contrib.admin',
    'easydata',
)
\end{lstlisting}

Cuando hayamos incluido la aplicación \textit{EasyData/Django} en nuestro
proyecto Django, lo siguiente que deberemos hacer será sincronizar la base de
datos, ya que la aplicación \textit{EasyData/Django} hace uso de sus propias
tablas para almacenar la información respecto a los namespaces, modelos y fields
existentes. Para ello, como debería de hacerse con cualquier aplicación,
deberemos ejecutar el siguiente comando en el directorio donde se encuentre el
manage.py de nuestro proyecto:

\begin{lstlisting}[frame=L, language=bash, basicstyle=\footnotesize]
$> python manage.py syncdb
\end{lstlisting}
 
El siguiente paso de la instalación de la aplicación EasyData/Django en nuestro
proyecto Django, será añadir las urls propias de la aplicación a las urls de
nuestro proyecto, de forma que el proyecto pueda servir dichas urls. Para
realizar esto, añada la siguiente línea al fichero url.py de su proyecto:

\begin{lstlisting}[frame=L, language=Python, basicstyle=\footnotesize]
url(r'^easydata/', include('easydata.urls'))
\end{lstlisting}

El nombre asignado de easydata para la url es opcional, de tal forma que se
puede modificar por otro que se adecue mejor a las necesidades del usuario.

Como último paso, la aplicación EasyData/Django usa staticfiles para servir los
ficheros estáticos de la aplicación, por lo que en nuestro proyecto deberá de
estar soportado. Para cargar los ficheros estáticos de la aplicación en nuestro
proyecto, se debe de lanzar el siguiente comando desde nuestro directorio del
proyecto:

\begin{lstlisting}[frame=L, language=bash, basicstyle=\footnotesize]
python manage.py collectstatic
\end{lstlisting}

Para más información acerca de los ficheros estáticos y de cómo se sirven estos
en un proyecto Django, visite la
\href{https://docs.djangoproject.com/en/dev/howto/static-files/}{documentación oficial}.

Por último, se informa al usuario que la aplicación EasyData/Django hace uso del
sistema de usuarios y de roles de Django para el control de permisos a las
vistas de configuración de la publicación de datos, restringiendo el acceso a
todo aquel usuario que no posea los permisos de super usuario de Django. Por lo
que si desea crear un usuario superusuario de Django, deberá hacerlo mediante el
siguiente comando en la terminal:

\begin{lstlisting}[frame=L, language=bash, basicstyle=\footnotesize]
python manage.py createsuperuser
\end{lstlisting}

El asistente se encargará de solicitarle los datos necesarios para la creación
del usuario. Si por otro lado, desea poder utilizar un usuario existente que no
está marcado como superusuario, únicamente deberá marcar a True los campos
\textit{is\_staff} y \textit{is\_superuser}.

\subsection{Otras configuraciones}

Con los pasos explicados anteriormente sería suficiente para que la aplicación
pudiera funcionar correctamente, pero existen otros parámetros de configuración
que pueden modificarse, para personalizar la misma a las necesidades del usuario.

\subsubsection{Cabecera de URLs}
\label{sec:cabeceraurl}

La aplicación EasyData utiliza el framework Sites de Django para obtener la
cabecera de las URLs (dominio), pero si el usuario lo desea, puede especificar
una función alternativa para calcular las cabeceras. Para ello, deberá
implementar una función propia que devuelva una cadena de texto con la cabecera
de las URLs, bajo el nombre \textit{url\_header} y especificar en el settings de
Django el módulo donde se encuentra dicha función mediante la variable
\mbox{\textit{EASYDATA\_URL\_HEADER}}, de manera que EasyData sepa donde buscar
dicha función.

\subsubsection{URIs de modelos}

Por otro lado, la aplicación EasyData de Django, genera una URI única para cada
una de las instancias de los modelos del proyecto Django donde ha sido
instalado, la cual hace referencia a el fichero RDF con los datos de dicha
instancia con el mapeo configurado.

En vez de usar la URI que hace referencia al fichero RDF con los datos de la
instancia, al usuario le podría interesar que se utilizase otro tipo de URI para
identificar a las instancias de un determinado modelo (como por ejemplo la URL
donde de la web donde se lista la información de dicha instancia), por lo que
existe la posibilidad de que el usuario especifique la URL para un determinado
modelo implementando el método \textit{easydata\_generate\_url}, el cual
devolverá el path absoluto (excepto la cabecera de la URL que utilizará la
función explicada en \ref{sec:cabeceraurl}).

\subsubsection{URIs para D2Rq}

La aplicación EasyData/Django generará una URI por defecto para el fichero de
configuración de D2Rq, aunque también será posible configurar esta URI para que
utilice una especificada por el usuario. Para modificar la URI que utilizará la
aplicación para D2Rq, se deberá crear en el modelo que se desee modificar la URI
una variable de nombre \textit{easydata\_url\_d2rq} la cual sea de tipo string y
contenga en el formato utilizado por D2Rq la plantilla que se utilizará para
generar las URI del modelo.

\subsubsection{Limitar ficheros RDF}

La aplicación EasyData/Django por defecto limitará a un número máximo de 50
instancias a la hora de exportar los datos de estas a través de RDF, de forma
que se evite sobrecargar al servidor debido a modelos con grandes volúmenes de
datos. De igual forma, este número máximo de instancias puede ser configurado
por el usuario administrador, añadiendo en el fichero settings del proyecto la
variable \textit{EASYDATA\_PUBLISH\_LIMIT}, el cual recibirá un número entero
que hará referencia al número máximo de instancias a generar en los ficheros
RDF.

\subsubsection{Creación de nuevos template tags para RDFa y Microdata}

A la hora de realizar la publicación de los datos en formato HTML, se utiliza en
la aplicación los template tags creados para tal fin, donde cada uno de ellos
utilizan diferentes etiquetas HTML para realizar la publicación. Puede que estas
etiquetas no sean precisamente las que el usuario necesita para realizar la
publicación, por ello, se ofrece al usuario la posibilidad de crear nuevos
template tags que usen etiquetas HTML diferentes.

Los template tags para publicar los datos, únicamente deben de devolver el
resultado de invocar a las funciones \textit{generate\_html\_rdfa} y
\textit{generate\_html\_microdata}, ubicadas en los ficheros easydata\_rdfa.py y
easydata\_microdata.py respectivamente, del directorio templatetags del proyecto.
Estas funciones tienen las siguientes cabeceras:

\begin{lstlisting}[frame=L, language=Python, basicstyle=\footnotesize]
generate_html_rdfa(instance, tag1, tag2, content)

generate_html_microdata(instance, tag1, tag2, content)
\end{lstlisting}

Donde cada uno de los atributos que reciben las funciones hacen referencia a:
\begin{itemize}
    \item \textbf{instance}: es la instancia del modelo de la que se desea
        generar el HTML con los datos.
    \item \textbf{tag1}: es la etiqueta HTML que se desea usar para incluir la
        información principal de la instancia.
    \item \textbf{tag2}: es la etiqueta HTML que se desea usar para incluir cada
        uno de los datos de la instancia.
    \item \textbf{content}: recibe True o False, lo que indica que para mostrar
        los datos se haga uso o no, del atributo content de las etiquetas.
\end{itemize}



\section{Procedimientos de operación y nivel de servicio}

No existen procedimientos necesarios que deban de llevarse a cabo para asegurar
el correcto funcionamiento de la aplicación, a excepción de los descritos
anteriormente para la puesta en marcha del mismo. Si bien, como ocurre con
cualquier proyecto software, se recomienda la realización periódica de back-ups,
para prevenir la pérdida de datos, y en el caso de esta aplicación, la perdida
de la configuración del mapeo de los modelos.

Por otro lado, se advierte al usuario que se está trabajando con una aplicación
para la publicación de datos de forma controlada de bases de datos de proyectos
Django, lo cual significa, que debe de prestarse excesivo cuidado a la hora de
indicar los datos que van a mostrarse públicamente, en referencia a los
problemas que pudiesen acarrear la publicación de ciertos datos sensibles de a
ser publicados. Esta responsabilidad recae únicamente sobre el usuario
administrador de la aplicación, el cual deberá de prestar especial atención al
apartado de visibilidad de los datos.


\section{Pruebas de implantación}

Una vez haya realizado todos los pasos descritos en el apartado
\ref{sec:procinsta}, para comprobar que la aplicación ha sido instalada
correctamente dentro de su proyecto Django, deberá de acceder a la siguiente
url, donde se le mostrará una pantalla de bienvenida a la aplicación. Sino
consigue acceder a la aplicación, revise los pasos descritos anteriormente.

\begin{center}
\textit{http://base\_url/easydata/}
\end{center}

Donde:
\begin{itemize}
    \item Deberá sustituir el protocolo http, por cualquier otro protocolo
    diferente que utilice, como por ejemplo comunicación segura mediante SSL.
    \item base\_url hace referencia a la dirección base de su proyecto. Si
    ejecuta por defecto runserver de django, esta dirección sería 127.0.0.1:8000.
    \item El nombre de easydata es el que hemos asignado en el paso anterior
    del tutorial, si se hubiera asignado otro distinto, entonces este debería
    modificarse por el nombre asignado en el fichero urls.py.
\end{itemize}

Suponiendo que estemos usando el protocolo http, con la dirección IP
127.0.0.1:8000 y el nombre para la url indicado en la documentación, la
dirección a la que deberíamos de acceder, sería la siguiente:

\begin{center}
\textit{http://127.0.0.1:8000/easydata/}
\end{center}
