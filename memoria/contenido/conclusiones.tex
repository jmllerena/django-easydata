% ------------------------------------------------------------------------------
% Este fichero es parte de la plantilla LaTeX para la realización de Proyectos
% Final de Grado, protegido bajo los términos de la licencia GFDL.
% Para más información, la licencia completa viene incluida en el
% fichero fdl-1.3.tex

% Copyright (C) 2012 SPI-FM. Universidad de Cádiz
% ------------------------------------------------------------------------------

En este último capítulo se detallan las lecciones aprendidas tras el desarrollo
del presente proyecto y se identifican las posibles oportunidades de mejora
sobre el software desarrollado.

\section{Objetivos}

A continuación, se recoge una valoración de los objetivos planteados
inicialmente en la sección \ref{sec:objetivos} que debería de recoger el
proyecto y los objetivos que se han alcanzado una vez finalizado el proyecto:
\begin{itemize}
    \item Se han implementado una serie de modelos de Django los cuales se
        encargan de almacenar tanto la información de los distintos namespaces
        cargados en la aplicación junto con las entidades y propiedades que
        componen a estos, como la información acerca de los diferentes modelos
        que componen a las aplicaciones del proyecto Django junto con los fields
        de los mismos. De esta forma se satisface el primer subobjetivo
        planteado en el objetivo \textbf{OBJ-001}.
    \item Se ha desarrollado un script y agregado este al manage.py de Django,
    	que se encarga de realizar la resolución de los distintos modelos que
        componen al proyecto, así como los fields (atributos y relaciones) de
        cada uno de estos. Con esto se satisface el segundo subobjetivo
        planteado en el objetivo \textbf{OBJ-001}.
    \item Además, el procedimiento anterior, también permite actualizar la
        especificación de los modelos y sus fields una vez que estos ya han sido
        cargados, cuando se realicen cambios en los mismos. Como por ejemplo,
        que se agregue o elimine algún modelo o field. De esta forma, se amplía
        la funcionalidad descrita por el objetivo \textbf{OBJ-001}.
    \item Se ha desarrollado un apartado donde el usuario puede añadir nuevos
        namespaces u ontologías a la aplicación. Además, dicho apartado también
        permite eliminar los namespaces incluidos, y actualizar la
        especificación de los mismos. De esta forma, se satisface el primer
        subobjetivo del objetivo \textbf{OBJ-002}.
    \item Se ha desarrollado otro apartado donde se permite al usuario realizar
        la correspondencia entre los distintos modelos y fields de nuestro
        proyecto, con los diferentes vocabularios que se han cargado en la
        aplicación. De esta forma se satisface los subobjetivos segundo y
        tercero del objetivo \textbf{OBJ-002}.
    \item También se ha desarrollado un apartado donde el usuario puede realizar
    	la configuración de la visibilidad de cada uno de los modelos y fields
    	que existen en el proyecto Django, de forma que no se permita publicar
        información no deseada. Con esto se satisface el cuarto subobjetivo del
        objetivo \textbf{OBJ-002}.
    \item Se han creado además, dos nuevos apartados donde se puede generar un
        fichero de configuración para utilizar con la plataforma D2Rq y también
        permite descargar un gráfico con la configuración que se ha realizado
        sobre los modelos de nuestro proyecto. Con esto se mejora la
        especificación del objetivo \textbf{OBJ-002}.
    \item Se han desarrollado herramientas que permiten la publicación de los
        datos de los modelos del proyecto Django, en multitud de formatos, como
        pueden ser RDF/XML, RDF/Ntriples, RDF/Turtle, RDFa o Microdata, además
        de herramientas para la generación de enlaces a los datos de las
        instancias de los modelos e inserción de información en plantillas
        Django. De esta forma, se satisface el objetivo \textbf{OBJ-003}.
\end{itemize}

% Este apartado debe resumir los objetivos generales y específicos alcanzados, relacionándolos con todo lo descrito en el capítulo de introducción.\\

\section{Lecciones aprendidas}

A continuación se detallan las buenas prácticas adquiridas, tanto tecnológicas
como procedimentales. Por ello, podemos decir que a nivel tecnológico, se ha
adquirido un mayor dominio y conocimiento del framework de desarrollo Django, de
su funcionamiento interno y de las diferentes herramientas que este presta al
usuario. Además, también se han obtenido nuevos conocimientos en tecnologías como
SPARQL, D2Rq, RDF y web semántica. En general, al haber sido realizado todo bajo
el lenguaje de programación Python, se ha obtenido un mayor dominio de este
lenguaje, se han conocido nuevas librerías disponibles en el
\textit{Python Package Index} y características que ofrece el lenguaje, como
puede ser la introspección. Además, en el ámbito del desarrollo web, también he
tenido que hacer uso de otras tecnologías, ya conocidas anteriormente, pero que
me ha ayudado a ampliar mis conocimientos sobre ellas, así como el uso de otros
frameworks, para el contenido tanto CSS como JavaScript.

Además de todos los conocimientos tecnológicos adquiridos, también se han
obtenido buenas prácticas a la hora de programar y organizar el trabajo. Esto se
ha conseguido siguiendo la guía de estilos PEP8 y la herramienta PyLint a la
hora de realizar la implementación del proyecto, lo cual es importante, ya que
asegura que el código será mantenible y comprensible por cualquier usuario que
esté habituado a trabajar siguiendo estas normas de estilo.

También se han puesto en práctica los conocimientos adquiridos tanto en mis
estudios de Ingeniería Técnica en Informática de Sistemas como en Ingeniería
Informática, ya que estos me han sido necesarios para la elaboración del
proyecto, tanto en el apartado de análisis y diseño de requisitos, prototipado,
análisis de riesgos, como en el apartado de implementación, haciendo uso de las
buenas prácticas adquiridas, como por ejemplo el uso de patrones de diseño.

Por último, a nivel organizativo, se ha seguido una guía metodológica haciendo
uso de un lenguaje de modelado de procesos, donde se describen y priorizan cada
una de las etapas del proyecto. Se ha realizado un análisis y diseño del
proyecto, así como una planificación de riesgos y temporal del proyecto. En
dicho estudio temporal del proyecto, el cual se puede ver el resultado en el
diagrama de Gantt del apartado de análisis (Figura \ref{fig:GanttInicial}), se
le asignaba al proyecto una duración estimada de 12 meses, comenzado el
desarrollo del mismo en Octubre de 2012, teniendo en cuenta que la dedicación al
mismo no sería total y el resto de riesgos que pudiesen surgir. De esta forma,
se ha aprendido la importancia de la gestión de los proyectos, de la estimación
temporal y del esfuerzo de los mismos y de la realización de un análisis y
diseño del mismo, de tal forma que se verá reflejado en una mejor administración
de los recursos y cumplimiento de plazos.

% Este apartado debe recoger una comparación cuantitativa del tiempo y el esfuerzo realmente invertido frente al estimado y planificado. Estos datos pueden recogerse del sistema de gestión de tareas empleado para el seguimiento del proyecto. Es mejor resumir cuantitativamente el tiempo y esfuerzo dedicados al proyecto a lo largo de su desarrollo y medido de esta forma que escribir un sencillo ``he trabajado mucho en este proyecto''.


\section{Trabajo futuro}

Una vez concluido el proyecto y obtenida la primera versión del software
EasyData/Django, a raíz de la experiencia obtenida a lo largo del desarrollo del
mismo, se han podido observar una serie de mejoras o ampliaciones, en cuya
dirección podría ir orientado el trabajo futuro para la aplicación. Estas
ampliaciones aportarían a la aplicación un mayor valor para el usuario,
aumentando las funcionalidades de la misma.

A continuación se muestra una lista de posibles mejoras en las que podría ir
encaminado el trabajo futuro:
\begin{itemize}
    \item Posibilidad de poder realizar más de una configuración simultánea de
        los modelos de la aplicación, con los namespaces cargados en la misma.
        De este modo se ofrecería al usuario mayor versatilidad a la hora de
        publicar sus datos en la web.
    \item Incluir un endpoint de SPARQL propio, que funcione desde la misma
    	aplicación de Django, que sustituya al fichero D2Rq que se genera
    	actualmente, quedando todo integrado en la misma aplicación. De igual
    	forma, si se desarrollase este módulo, se podría considerar la
    	posibilidad de que no sólo pudiesen realizarse consultas, sino también
        inserciones y modificaciones de los datos.
    \item Ampliar los formatos de exportación de datos respecto de los que
        pueden usarse actualmente (RDF, RDFa y Microdata), usando otros formatos
        existentes como por ejemplo JSON-LD además de ampliar los existentes.
    \item Ampliar la disponibilidad de la aplicación EasyData (actualmente
        existente para Ruby on Rails y Django) a otros frameworks de desarrollo
        web que sean ampliamente utilizados actualmente.
\end{itemize}

Si bien, todas las posibles mejoras no requieren un mismo esfuerzo, ya que
mejoras como las de la creación de un módulo propio para la realización de
consultas haciendo uso del lenguaje SPARQL es bastante ambicioso y complejo, el
cual requiere de un gran trabajo, tanto de investigación como de desarrollo, ya
que habría que estudiar primeramente como funciona el ORM de Django, para
posteriormente adaptar las consultas SPARQL a este. Además, si se tratase la
posibilidad de realizar inserciones y modificaciones en la base de datos, habría
que gestionar de alguna forma los permisos de los usuarios para realizar las
mismas.

Por otro lado, mejoras como la exportación de la aplicación a otros frameworks
de desarrollo web, habría que estudiar entre las posibilidades de realizar una
nueva aplicación para el framework en concreto, o la posibilidad de adaptar la
existente a cualquier framework. O al menos agrupar estos por lenguajes de
programación, como por ejemplo, en el caso de Python, existen varios frameworks
de desarrollo web además de Django, como puede ser Web2Py, Zope, etc... por lo
que también se podría estudiar la posibilidad de desarrollar un paquete de Python
que sirviese para la mayoría de los frameworks web de Python.

% En esta sección, se presentan las diversas áreas u oportunidades de mejora detectadas durante el desarrollo del proyecto y que podrán ser abarcadas en futuras versiones del software.\\

% Los elementos aquí descritos deben estar en relación con lo relatado en el apartado de objetivos y alcance del proyecto descritos en la introducción.\\

